\documentclass[11pt]{article}

    \usepackage[breakable]{tcolorbox}
    \usepackage{parskip} % Stop auto-indenting (to mimic markdown behaviour)
    
    \usepackage{iftex}
    \ifPDFTeX
    	\usepackage[T1]{fontenc}
    	\usepackage{mathpazo}
    \else
    	\usepackage{fontspec}
    \fi

    % Basic figure setup, for now with no caption control since it's done
    % automatically by Pandoc (which extracts ![](path) syntax from Markdown).
    \usepackage{graphicx}
    % Maintain compatibility with old templates. Remove in nbconvert 6.0
    \let\Oldincludegraphics\includegraphics
    % Ensure that by default, figures have no caption (until we provide a
    % proper Figure object with a Caption API and a way to capture that
    % in the conversion process - todo).
    \usepackage{caption}
    \DeclareCaptionFormat{nocaption}{}
    \captionsetup{format=nocaption,aboveskip=0pt,belowskip=0pt}

    \usepackage[Export]{adjustbox} % Used to constrain images to a maximum size
    \adjustboxset{max size={0.9\linewidth}{0.9\paperheight}}
    \usepackage{float}
    \floatplacement{figure}{H} % forces figures to be placed at the correct location
    \usepackage{xcolor} % Allow colors to be defined
    \usepackage{enumerate} % Needed for markdown enumerations to work
    \usepackage{geometry} % Used to adjust the document margins
    \usepackage{amsmath} % Equations
    \usepackage{amssymb} % Equations
    \usepackage{textcomp} % defines textquotesingle
    % Hack from http://tex.stackexchange.com/a/47451/13684:
    \AtBeginDocument{%
        \def\PYZsq{\textquotesingle}% Upright quotes in Pygmentized code
    }
    \usepackage{upquote} % Upright quotes for verbatim code
    \usepackage{eurosym} % defines \euro
    \usepackage[mathletters]{ucs} % Extended unicode (utf-8) support
    \usepackage{fancyvrb} % verbatim replacement that allows latex
    \usepackage{grffile} % extends the file name processing of package graphics 
                         % to support a larger range
    \makeatletter % fix for grffile with XeLaTeX
    \def\Gread@@xetex#1{%
      \IfFileExists{"\Gin@base".bb}%
      {\Gread@eps{\Gin@base.bb}}%
      {\Gread@@xetex@aux#1}%
    }
    \makeatother

    % The hyperref package gives us a pdf with properly built
    % internal navigation ('pdf bookmarks' for the table of contents,
    % internal cross-reference links, web links for URLs, etc.)
    \usepackage{hyperref}
    % The default LaTeX title has an obnoxious amount of whitespace. By default,
    % titling removes some of it. It also provides customization options.
    \usepackage{titling}
    \usepackage{longtable} % longtable support required by pandoc >1.10
    \usepackage{booktabs}  % table support for pandoc > 1.12.2
    \usepackage[inline]{enumitem} % IRkernel/repr support (it uses the enumerate* environment)
    \usepackage[normalem]{ulem} % ulem is needed to support strikethroughs (\sout)
                                % normalem makes italics be italics, not underlines
    \usepackage{mathrsfs}
    

    
    % Colors for the hyperref package
    \definecolor{urlcolor}{rgb}{0,.145,.698}
    \definecolor{linkcolor}{rgb}{.71,0.21,0.01}
    \definecolor{citecolor}{rgb}{.12,.54,.11}

    % ANSI colors
    \definecolor{ansi-black}{HTML}{3E424D}
    \definecolor{ansi-black-intense}{HTML}{282C36}
    \definecolor{ansi-red}{HTML}{E75C58}
    \definecolor{ansi-red-intense}{HTML}{B22B31}
    \definecolor{ansi-green}{HTML}{00A250}
    \definecolor{ansi-green-intense}{HTML}{007427}
    \definecolor{ansi-yellow}{HTML}{DDB62B}
    \definecolor{ansi-yellow-intense}{HTML}{B27D12}
    \definecolor{ansi-blue}{HTML}{208FFB}
    \definecolor{ansi-blue-intense}{HTML}{0065CA}
    \definecolor{ansi-magenta}{HTML}{D160C4}
    \definecolor{ansi-magenta-intense}{HTML}{A03196}
    \definecolor{ansi-cyan}{HTML}{60C6C8}
    \definecolor{ansi-cyan-intense}{HTML}{258F8F}
    \definecolor{ansi-white}{HTML}{C5C1B4}
    \definecolor{ansi-white-intense}{HTML}{A1A6B2}
    \definecolor{ansi-default-inverse-fg}{HTML}{FFFFFF}
    \definecolor{ansi-default-inverse-bg}{HTML}{000000}

    % commands and environments needed by pandoc snippets
    % extracted from the output of `pandoc -s`
    \providecommand{\tightlist}{%
      \setlength{\itemsep}{0pt}\setlength{\parskip}{0pt}}
    \DefineVerbatimEnvironment{Highlighting}{Verbatim}{commandchars=\\\{\}}
    % Add ',fontsize=\small' for more characters per line
    \newenvironment{Shaded}{}{}
    \newcommand{\KeywordTok}[1]{\textcolor[rgb]{0.00,0.44,0.13}{\textbf{{#1}}}}
    \newcommand{\DataTypeTok}[1]{\textcolor[rgb]{0.56,0.13,0.00}{{#1}}}
    \newcommand{\DecValTok}[1]{\textcolor[rgb]{0.25,0.63,0.44}{{#1}}}
    \newcommand{\BaseNTok}[1]{\textcolor[rgb]{0.25,0.63,0.44}{{#1}}}
    \newcommand{\FloatTok}[1]{\textcolor[rgb]{0.25,0.63,0.44}{{#1}}}
    \newcommand{\CharTok}[1]{\textcolor[rgb]{0.25,0.44,0.63}{{#1}}}
    \newcommand{\StringTok}[1]{\textcolor[rgb]{0.25,0.44,0.63}{{#1}}}
    \newcommand{\CommentTok}[1]{\textcolor[rgb]{0.38,0.63,0.69}{\textit{{#1}}}}
    \newcommand{\OtherTok}[1]{\textcolor[rgb]{0.00,0.44,0.13}{{#1}}}
    \newcommand{\AlertTok}[1]{\textcolor[rgb]{1.00,0.00,0.00}{\textbf{{#1}}}}
    \newcommand{\FunctionTok}[1]{\textcolor[rgb]{0.02,0.16,0.49}{{#1}}}
    \newcommand{\RegionMarkerTok}[1]{{#1}}
    \newcommand{\ErrorTok}[1]{\textcolor[rgb]{1.00,0.00,0.00}{\textbf{{#1}}}}
    \newcommand{\NormalTok}[1]{{#1}}
    
    % Additional commands for more recent versions of Pandoc
    \newcommand{\ConstantTok}[1]{\textcolor[rgb]{0.53,0.00,0.00}{{#1}}}
    \newcommand{\SpecialCharTok}[1]{\textcolor[rgb]{0.25,0.44,0.63}{{#1}}}
    \newcommand{\VerbatimStringTok}[1]{\textcolor[rgb]{0.25,0.44,0.63}{{#1}}}
    \newcommand{\SpecialStringTok}[1]{\textcolor[rgb]{0.73,0.40,0.53}{{#1}}}
    \newcommand{\ImportTok}[1]{{#1}}
    \newcommand{\DocumentationTok}[1]{\textcolor[rgb]{0.73,0.13,0.13}{\textit{{#1}}}}
    \newcommand{\AnnotationTok}[1]{\textcolor[rgb]{0.38,0.63,0.69}{\textbf{\textit{{#1}}}}}
    \newcommand{\CommentVarTok}[1]{\textcolor[rgb]{0.38,0.63,0.69}{\textbf{\textit{{#1}}}}}
    \newcommand{\VariableTok}[1]{\textcolor[rgb]{0.10,0.09,0.49}{{#1}}}
    \newcommand{\ControlFlowTok}[1]{\textcolor[rgb]{0.00,0.44,0.13}{\textbf{{#1}}}}
    \newcommand{\OperatorTok}[1]{\textcolor[rgb]{0.40,0.40,0.40}{{#1}}}
    \newcommand{\BuiltInTok}[1]{{#1}}
    \newcommand{\ExtensionTok}[1]{{#1}}
    \newcommand{\PreprocessorTok}[1]{\textcolor[rgb]{0.74,0.48,0.00}{{#1}}}
    \newcommand{\AttributeTok}[1]{\textcolor[rgb]{0.49,0.56,0.16}{{#1}}}
    \newcommand{\InformationTok}[1]{\textcolor[rgb]{0.38,0.63,0.69}{\textbf{\textit{{#1}}}}}
    \newcommand{\WarningTok}[1]{\textcolor[rgb]{0.38,0.63,0.69}{\textbf{\textit{{#1}}}}}
    
    
    % Define a nice break command that doesn't care if a line doesn't already
    % exist.
    \def\br{\hspace*{\fill} \\* }
    % Math Jax compatibility definitions
    \def\gt{>}
    \def\lt{<}
    \let\Oldtex\TeX
    \let\Oldlatex\LaTeX
    \renewcommand{\TeX}{\textrm{\Oldtex}}
    \renewcommand{\LaTeX}{\textrm{\Oldlatex}}
    % Document parameters
    % Document title
    \title{[LA]Chapter01\_Summary\_201838205\_김나연 - 복사본}
    
    
    
    
    
% Pygments definitions
\makeatletter
\def\PY@reset{\let\PY@it=\relax \let\PY@bf=\relax%
    \let\PY@ul=\relax \let\PY@tc=\relax%
    \let\PY@bc=\relax \let\PY@ff=\relax}
\def\PY@tok#1{\csname PY@tok@#1\endcsname}
\def\PY@toks#1+{\ifx\relax#1\empty\else%
    \PY@tok{#1}\expandafter\PY@toks\fi}
\def\PY@do#1{\PY@bc{\PY@tc{\PY@ul{%
    \PY@it{\PY@bf{\PY@ff{#1}}}}}}}
\def\PY#1#2{\PY@reset\PY@toks#1+\relax+\PY@do{#2}}

\expandafter\def\csname PY@tok@w\endcsname{\def\PY@tc##1{\textcolor[rgb]{0.73,0.73,0.73}{##1}}}
\expandafter\def\csname PY@tok@c\endcsname{\let\PY@it=\textit\def\PY@tc##1{\textcolor[rgb]{0.25,0.50,0.50}{##1}}}
\expandafter\def\csname PY@tok@cp\endcsname{\def\PY@tc##1{\textcolor[rgb]{0.74,0.48,0.00}{##1}}}
\expandafter\def\csname PY@tok@k\endcsname{\let\PY@bf=\textbf\def\PY@tc##1{\textcolor[rgb]{0.00,0.50,0.00}{##1}}}
\expandafter\def\csname PY@tok@kp\endcsname{\def\PY@tc##1{\textcolor[rgb]{0.00,0.50,0.00}{##1}}}
\expandafter\def\csname PY@tok@kt\endcsname{\def\PY@tc##1{\textcolor[rgb]{0.69,0.00,0.25}{##1}}}
\expandafter\def\csname PY@tok@o\endcsname{\def\PY@tc##1{\textcolor[rgb]{0.40,0.40,0.40}{##1}}}
\expandafter\def\csname PY@tok@ow\endcsname{\let\PY@bf=\textbf\def\PY@tc##1{\textcolor[rgb]{0.67,0.13,1.00}{##1}}}
\expandafter\def\csname PY@tok@nb\endcsname{\def\PY@tc##1{\textcolor[rgb]{0.00,0.50,0.00}{##1}}}
\expandafter\def\csname PY@tok@nf\endcsname{\def\PY@tc##1{\textcolor[rgb]{0.00,0.00,1.00}{##1}}}
\expandafter\def\csname PY@tok@nc\endcsname{\let\PY@bf=\textbf\def\PY@tc##1{\textcolor[rgb]{0.00,0.00,1.00}{##1}}}
\expandafter\def\csname PY@tok@nn\endcsname{\let\PY@bf=\textbf\def\PY@tc##1{\textcolor[rgb]{0.00,0.00,1.00}{##1}}}
\expandafter\def\csname PY@tok@ne\endcsname{\let\PY@bf=\textbf\def\PY@tc##1{\textcolor[rgb]{0.82,0.25,0.23}{##1}}}
\expandafter\def\csname PY@tok@nv\endcsname{\def\PY@tc##1{\textcolor[rgb]{0.10,0.09,0.49}{##1}}}
\expandafter\def\csname PY@tok@no\endcsname{\def\PY@tc##1{\textcolor[rgb]{0.53,0.00,0.00}{##1}}}
\expandafter\def\csname PY@tok@nl\endcsname{\def\PY@tc##1{\textcolor[rgb]{0.63,0.63,0.00}{##1}}}
\expandafter\def\csname PY@tok@ni\endcsname{\let\PY@bf=\textbf\def\PY@tc##1{\textcolor[rgb]{0.60,0.60,0.60}{##1}}}
\expandafter\def\csname PY@tok@na\endcsname{\def\PY@tc##1{\textcolor[rgb]{0.49,0.56,0.16}{##1}}}
\expandafter\def\csname PY@tok@nt\endcsname{\let\PY@bf=\textbf\def\PY@tc##1{\textcolor[rgb]{0.00,0.50,0.00}{##1}}}
\expandafter\def\csname PY@tok@nd\endcsname{\def\PY@tc##1{\textcolor[rgb]{0.67,0.13,1.00}{##1}}}
\expandafter\def\csname PY@tok@s\endcsname{\def\PY@tc##1{\textcolor[rgb]{0.73,0.13,0.13}{##1}}}
\expandafter\def\csname PY@tok@sd\endcsname{\let\PY@it=\textit\def\PY@tc##1{\textcolor[rgb]{0.73,0.13,0.13}{##1}}}
\expandafter\def\csname PY@tok@si\endcsname{\let\PY@bf=\textbf\def\PY@tc##1{\textcolor[rgb]{0.73,0.40,0.53}{##1}}}
\expandafter\def\csname PY@tok@se\endcsname{\let\PY@bf=\textbf\def\PY@tc##1{\textcolor[rgb]{0.73,0.40,0.13}{##1}}}
\expandafter\def\csname PY@tok@sr\endcsname{\def\PY@tc##1{\textcolor[rgb]{0.73,0.40,0.53}{##1}}}
\expandafter\def\csname PY@tok@ss\endcsname{\def\PY@tc##1{\textcolor[rgb]{0.10,0.09,0.49}{##1}}}
\expandafter\def\csname PY@tok@sx\endcsname{\def\PY@tc##1{\textcolor[rgb]{0.00,0.50,0.00}{##1}}}
\expandafter\def\csname PY@tok@m\endcsname{\def\PY@tc##1{\textcolor[rgb]{0.40,0.40,0.40}{##1}}}
\expandafter\def\csname PY@tok@gh\endcsname{\let\PY@bf=\textbf\def\PY@tc##1{\textcolor[rgb]{0.00,0.00,0.50}{##1}}}
\expandafter\def\csname PY@tok@gu\endcsname{\let\PY@bf=\textbf\def\PY@tc##1{\textcolor[rgb]{0.50,0.00,0.50}{##1}}}
\expandafter\def\csname PY@tok@gd\endcsname{\def\PY@tc##1{\textcolor[rgb]{0.63,0.00,0.00}{##1}}}
\expandafter\def\csname PY@tok@gi\endcsname{\def\PY@tc##1{\textcolor[rgb]{0.00,0.63,0.00}{##1}}}
\expandafter\def\csname PY@tok@gr\endcsname{\def\PY@tc##1{\textcolor[rgb]{1.00,0.00,0.00}{##1}}}
\expandafter\def\csname PY@tok@ge\endcsname{\let\PY@it=\textit}
\expandafter\def\csname PY@tok@gs\endcsname{\let\PY@bf=\textbf}
\expandafter\def\csname PY@tok@gp\endcsname{\let\PY@bf=\textbf\def\PY@tc##1{\textcolor[rgb]{0.00,0.00,0.50}{##1}}}
\expandafter\def\csname PY@tok@go\endcsname{\def\PY@tc##1{\textcolor[rgb]{0.53,0.53,0.53}{##1}}}
\expandafter\def\csname PY@tok@gt\endcsname{\def\PY@tc##1{\textcolor[rgb]{0.00,0.27,0.87}{##1}}}
\expandafter\def\csname PY@tok@err\endcsname{\def\PY@bc##1{\setlength{\fboxsep}{0pt}\fcolorbox[rgb]{1.00,0.00,0.00}{1,1,1}{\strut ##1}}}
\expandafter\def\csname PY@tok@kc\endcsname{\let\PY@bf=\textbf\def\PY@tc##1{\textcolor[rgb]{0.00,0.50,0.00}{##1}}}
\expandafter\def\csname PY@tok@kd\endcsname{\let\PY@bf=\textbf\def\PY@tc##1{\textcolor[rgb]{0.00,0.50,0.00}{##1}}}
\expandafter\def\csname PY@tok@kn\endcsname{\let\PY@bf=\textbf\def\PY@tc##1{\textcolor[rgb]{0.00,0.50,0.00}{##1}}}
\expandafter\def\csname PY@tok@kr\endcsname{\let\PY@bf=\textbf\def\PY@tc##1{\textcolor[rgb]{0.00,0.50,0.00}{##1}}}
\expandafter\def\csname PY@tok@bp\endcsname{\def\PY@tc##1{\textcolor[rgb]{0.00,0.50,0.00}{##1}}}
\expandafter\def\csname PY@tok@fm\endcsname{\def\PY@tc##1{\textcolor[rgb]{0.00,0.00,1.00}{##1}}}
\expandafter\def\csname PY@tok@vc\endcsname{\def\PY@tc##1{\textcolor[rgb]{0.10,0.09,0.49}{##1}}}
\expandafter\def\csname PY@tok@vg\endcsname{\def\PY@tc##1{\textcolor[rgb]{0.10,0.09,0.49}{##1}}}
\expandafter\def\csname PY@tok@vi\endcsname{\def\PY@tc##1{\textcolor[rgb]{0.10,0.09,0.49}{##1}}}
\expandafter\def\csname PY@tok@vm\endcsname{\def\PY@tc##1{\textcolor[rgb]{0.10,0.09,0.49}{##1}}}
\expandafter\def\csname PY@tok@sa\endcsname{\def\PY@tc##1{\textcolor[rgb]{0.73,0.13,0.13}{##1}}}
\expandafter\def\csname PY@tok@sb\endcsname{\def\PY@tc##1{\textcolor[rgb]{0.73,0.13,0.13}{##1}}}
\expandafter\def\csname PY@tok@sc\endcsname{\def\PY@tc##1{\textcolor[rgb]{0.73,0.13,0.13}{##1}}}
\expandafter\def\csname PY@tok@dl\endcsname{\def\PY@tc##1{\textcolor[rgb]{0.73,0.13,0.13}{##1}}}
\expandafter\def\csname PY@tok@s2\endcsname{\def\PY@tc##1{\textcolor[rgb]{0.73,0.13,0.13}{##1}}}
\expandafter\def\csname PY@tok@sh\endcsname{\def\PY@tc##1{\textcolor[rgb]{0.73,0.13,0.13}{##1}}}
\expandafter\def\csname PY@tok@s1\endcsname{\def\PY@tc##1{\textcolor[rgb]{0.73,0.13,0.13}{##1}}}
\expandafter\def\csname PY@tok@mb\endcsname{\def\PY@tc##1{\textcolor[rgb]{0.40,0.40,0.40}{##1}}}
\expandafter\def\csname PY@tok@mf\endcsname{\def\PY@tc##1{\textcolor[rgb]{0.40,0.40,0.40}{##1}}}
\expandafter\def\csname PY@tok@mh\endcsname{\def\PY@tc##1{\textcolor[rgb]{0.40,0.40,0.40}{##1}}}
\expandafter\def\csname PY@tok@mi\endcsname{\def\PY@tc##1{\textcolor[rgb]{0.40,0.40,0.40}{##1}}}
\expandafter\def\csname PY@tok@il\endcsname{\def\PY@tc##1{\textcolor[rgb]{0.40,0.40,0.40}{##1}}}
\expandafter\def\csname PY@tok@mo\endcsname{\def\PY@tc##1{\textcolor[rgb]{0.40,0.40,0.40}{##1}}}
\expandafter\def\csname PY@tok@ch\endcsname{\let\PY@it=\textit\def\PY@tc##1{\textcolor[rgb]{0.25,0.50,0.50}{##1}}}
\expandafter\def\csname PY@tok@cm\endcsname{\let\PY@it=\textit\def\PY@tc##1{\textcolor[rgb]{0.25,0.50,0.50}{##1}}}
\expandafter\def\csname PY@tok@cpf\endcsname{\let\PY@it=\textit\def\PY@tc##1{\textcolor[rgb]{0.25,0.50,0.50}{##1}}}
\expandafter\def\csname PY@tok@c1\endcsname{\let\PY@it=\textit\def\PY@tc##1{\textcolor[rgb]{0.25,0.50,0.50}{##1}}}
\expandafter\def\csname PY@tok@cs\endcsname{\let\PY@it=\textit\def\PY@tc##1{\textcolor[rgb]{0.25,0.50,0.50}{##1}}}

\def\PYZbs{\char`\\}
\def\PYZus{\char`\_}
\def\PYZob{\char`\{}
\def\PYZcb{\char`\}}
\def\PYZca{\char`\^}
\def\PYZam{\char`\&}
\def\PYZlt{\char`\<}
\def\PYZgt{\char`\>}
\def\PYZsh{\char`\#}
\def\PYZpc{\char`\%}
\def\PYZdl{\char`\$}
\def\PYZhy{\char`\-}
\def\PYZsq{\char`\'}
\def\PYZdq{\char`\"}
\def\PYZti{\char`\~}
% for compatibility with earlier versions
\def\PYZat{@}
\def\PYZlb{[}
\def\PYZrb{]}
\makeatother


    % For linebreaks inside Verbatim environment from package fancyvrb. 
    \makeatletter
        \newbox\Wrappedcontinuationbox 
        \newbox\Wrappedvisiblespacebox 
        \newcommand*\Wrappedvisiblespace {\textcolor{red}{\textvisiblespace}} 
        \newcommand*\Wrappedcontinuationsymbol {\textcolor{red}{\llap{\tiny$\m@th\hookrightarrow$}}} 
        \newcommand*\Wrappedcontinuationindent {3ex } 
        \newcommand*\Wrappedafterbreak {\kern\Wrappedcontinuationindent\copy\Wrappedcontinuationbox} 
        % Take advantage of the already applied Pygments mark-up to insert 
        % potential linebreaks for TeX processing. 
        %        {, <, #, %, $, ' and ": go to next line. 
        %        _, }, ^, &, >, - and ~: stay at end of broken line. 
        % Use of \textquotesingle for straight quote. 
        \newcommand*\Wrappedbreaksatspecials {% 
            \def\PYGZus{\discretionary{\char`\_}{\Wrappedafterbreak}{\char`\_}}% 
            \def\PYGZob{\discretionary{}{\Wrappedafterbreak\char`\{}{\char`\{}}% 
            \def\PYGZcb{\discretionary{\char`\}}{\Wrappedafterbreak}{\char`\}}}% 
            \def\PYGZca{\discretionary{\char`\^}{\Wrappedafterbreak}{\char`\^}}% 
            \def\PYGZam{\discretionary{\char`\&}{\Wrappedafterbreak}{\char`\&}}% 
            \def\PYGZlt{\discretionary{}{\Wrappedafterbreak\char`\<}{\char`\<}}% 
            \def\PYGZgt{\discretionary{\char`\>}{\Wrappedafterbreak}{\char`\>}}% 
            \def\PYGZsh{\discretionary{}{\Wrappedafterbreak\char`\#}{\char`\#}}% 
            \def\PYGZpc{\discretionary{}{\Wrappedafterbreak\char`\%}{\char`\%}}% 
            \def\PYGZdl{\discretionary{}{\Wrappedafterbreak\char`\$}{\char`\$}}% 
            \def\PYGZhy{\discretionary{\char`\-}{\Wrappedafterbreak}{\char`\-}}% 
            \def\PYGZsq{\discretionary{}{\Wrappedafterbreak\textquotesingle}{\textquotesingle}}% 
            \def\PYGZdq{\discretionary{}{\Wrappedafterbreak\char`\"}{\char`\"}}% 
            \def\PYGZti{\discretionary{\char`\~}{\Wrappedafterbreak}{\char`\~}}% 
        } 
        % Some characters . , ; ? ! / are not pygmentized. 
        % This macro makes them "active" and they will insert potential linebreaks 
        \newcommand*\Wrappedbreaksatpunct {% 
            \lccode`\~`\.\lowercase{\def~}{\discretionary{\hbox{\char`\.}}{\Wrappedafterbreak}{\hbox{\char`\.}}}% 
            \lccode`\~`\,\lowercase{\def~}{\discretionary{\hbox{\char`\,}}{\Wrappedafterbreak}{\hbox{\char`\,}}}% 
            \lccode`\~`\;\lowercase{\def~}{\discretionary{\hbox{\char`\;}}{\Wrappedafterbreak}{\hbox{\char`\;}}}% 
            \lccode`\~`\:\lowercase{\def~}{\discretionary{\hbox{\char`\:}}{\Wrappedafterbreak}{\hbox{\char`\:}}}% 
            \lccode`\~`\?\lowercase{\def~}{\discretionary{\hbox{\char`\?}}{\Wrappedafterbreak}{\hbox{\char`\?}}}% 
            \lccode`\~`\!\lowercase{\def~}{\discretionary{\hbox{\char`\!}}{\Wrappedafterbreak}{\hbox{\char`\!}}}% 
            \lccode`\~`\/\lowercase{\def~}{\discretionary{\hbox{\char`\/}}{\Wrappedafterbreak}{\hbox{\char`\/}}}% 
            \catcode`\.\active
            \catcode`\,\active 
            \catcode`\;\active
            \catcode`\:\active
            \catcode`\?\active
            \catcode`\!\active
            \catcode`\/\active 
            \lccode`\~`\~ 	
        }
    \makeatother

    \let\OriginalVerbatim=\Verbatim
    \makeatletter
    \renewcommand{\Verbatim}[1][1]{%
        %\parskip\z@skip
        \sbox\Wrappedcontinuationbox {\Wrappedcontinuationsymbol}%
        \sbox\Wrappedvisiblespacebox {\FV@SetupFont\Wrappedvisiblespace}%
        \def\FancyVerbFormatLine ##1{\hsize\linewidth
            \vtop{\raggedright\hyphenpenalty\z@\exhyphenpenalty\z@
                \doublehyphendemerits\z@\finalhyphendemerits\z@
                \strut ##1\strut}%
        }%
        % If the linebreak is at a space, the latter will be displayed as visible
        % space at end of first line, and a continuation symbol starts next line.
        % Stretch/shrink are however usually zero for typewriter font.
        \def\FV@Space {%
            \nobreak\hskip\z@ plus\fontdimen3\font minus\fontdimen4\font
            \discretionary{\copy\Wrappedvisiblespacebox}{\Wrappedafterbreak}
            {\kern\fontdimen2\font}%
        }%
        
        % Allow breaks at special characters using \PYG... macros.
        \Wrappedbreaksatspecials
        % Breaks at punctuation characters . , ; ? ! and / need catcode=\active 	
        \OriginalVerbatim[#1,codes*=\Wrappedbreaksatpunct]%
    }
    \makeatother

    % Exact colors from NB
    \definecolor{incolor}{HTML}{303F9F}
    \definecolor{outcolor}{HTML}{D84315}
    \definecolor{cellborder}{HTML}{CFCFCF}
    \definecolor{cellbackground}{HTML}{F7F7F7}
    
    % prompt
    \makeatletter
    \newcommand{\boxspacing}{\kern\kvtcb@left@rule\kern\kvtcb@boxsep}
    \makeatother
    \newcommand{\prompt}[4]{
        \ttfamily\llap{{\color{#2}[#3]:\hspace{3pt}#4}}\vspace{-\baselineskip}
    }
    

    
    % Prevent overflowing lines due to hard-to-break entities
    \sloppy 
    % Setup hyperref package
    \hypersetup{
      breaklinks=true,  % so long urls are correctly broken across lines
      colorlinks=true,
      urlcolor=urlcolor,
      linkcolor=linkcolor,
      citecolor=citecolor,
      }
    % Slightly bigger margins than the latex defaults
    
    \geometry{verbose,tmargin=1in,bmargin=1in,lmargin=1in,rmargin=1in}
    
    

\begin{document}
    
    \maketitle
    
    

    
    \hypertarget{chapter-01.-linear-equations-in-linear-algebra}{%
\subsection{\# Chapter 01. Linear Equations in Linear
Algebra}\label{chapter-01.-linear-equations-in-linear-algebra}}

\hypertarget{systems-of-linear-equations}{%
\subsection{1.1 Systems of Linear
Equations}\label{systems-of-linear-equations}}

\hypertarget{linear-equationuxc774uxb780}{%
\subsubsection{1.1.1 linear
equation이란}\label{linear-equationuxc774uxb780}}

\(a_{1}x_{1} + a_{2}x_{2}+...+a_{n}x_{n}=b\) \textgreater{} \(b\) 와
coefficient ``\(a_{1}...a_{n}\)''는 real 또는 complex number 이다.

\begin{quote}
\emph{Warning}\\
\(4x_{1} - 5x_{2} = x_{1}x_{2}\) 는 linear하지 않다.

\hypertarget{system-of-linear-equationsor-linear-systemuxc774uxb780}{%
\subsubsection{1.1.2 system of linear equations(or linear
system)이란}\label{system-of-linear-equationsor-linear-systemuxc774uxb780}}

\begin{itemize}
\tightlist
\item
  하나 또는 하나 이상의 linear equation의 모음.
\item
  EX) \(2x_{1} - x_{2} + 1.5x_{3} = 8\)\\
  \(x_{1} - 5x_{2} + 2x_{3} = -7\)
\end{itemize}
\end{quote}

\hypertarget{solution-of-linear-system}{%
\subsubsection{1.1.3 solution of linear
system}\label{solution-of-linear-system}}

\begin{quote}
\begin{enumerate}
\def\labelenumi{\arabic{enumi}.}
\tightlist
\item
  no solution
\item
  exactly one solution
\item
  infinitely many solutions
\end{enumerate}
\end{quote}

    \hypertarget{matrix-notation}{%
\subsubsection{1.1.4 Matrix Notation}\label{matrix-notation}}

\begin{itemize}
\item
  \textbf{linear system} \textgreater{} \(x_{1} -2x_{2} + x_{3} = 0\)\\
  \textgreater{} \$ 2x\_\{2\} - 8x\_\{3\} = 8\$\\
  \textgreater{} \(5x_{1} -5x_{3} = 10\)
\item
  \textbf{coefiicient matrix(or matrix of coefficients)}\\
  \textgreater{}
  \(\begin{bmatrix} 1 & -2 & 1 \\ 0 & 2 & -8 \\ 5 & 0 & -5 \end{bmatrix}\)
\item
  \textbf{augmented matrix}\\
  \textgreater{}
  \(\begin{bmatrix} 1 & -2 & 1 & 0 \\ 0 & 2 & -8 & 8\\ 5 & 0 & -5 & 10\end{bmatrix}\)
\end{itemize}

    \hypertarget{solving-a-linear-system}{%
\subsubsection{1.1.5 Solving a Linear
System}\label{solving-a-linear-system}}

\begin{quote}
\textbf{Elementary Row Operations} 1. Replacement 2. Interchange 3.
Scaling
\end{quote}

    \hypertarget{existence-and-uniqueness-questions}{%
\subsubsection{1.1.6 Existence and Uniqueness
Questions}\label{existence-and-uniqueness-questions}}

\begin{quote}
\textbf{linear system에 대한 두가지 질문} 1. Is the system
\textbf{consistent} : that is, does at least one solution \emph{exist}?
(one solution or infinitely many solution) 2. If a solution
\textbf{exists}, is the \emph{only one} : that is , is the solution
\emph{unique}?
\includegraphics{https://user-images.githubusercontent.com/53852102/95113427-68790480-077d-11eb-9fab-5d2c5c93f6ab.png}
\end{quote}

    \begin{center}\rule{0.5\linewidth}{\linethickness}\end{center}

\hypertarget{row-reduction-and-echelon-form}{%
\subsection{1.2 Row Reduction and Echelon
Form}\label{row-reduction-and-echelon-form}}

\hypertarget{definition-echelon-form}{%
\subsubsection{\texorpdfstring{\emph{Definition} : echelon
form}{Definition : echelon form}}\label{definition-echelon-form}}

\begin{quote}
a rectangular matrix가 \textbf{echelon form(또는 row echelon form)} 이면
아래와 같은 특성을 따른다.\\
1. 모든 nonzero row는 zero 행 위에 있다. 2. 각 행의 leading entry는 그
위의 leading entry에 비해 오른쪽에 위치한다. 3. leading entry 밑에 있는
columndm은 모두 zero이다. \textbf{reduced echelon form (or reduced row
echelon form, RREF)} 일 때는 4. leading entry 가 1이다. (; leading 1)\\
5. leading 1을 포함하는 각 열은 leading 1의 위아래가 0이어야 한다. *
row-reduced란 elementary row operations에 의해 변환된 것
\end{quote}

    \hypertarget{theorem-1-uniqueness-of-the-reduced-echelon-form}{%
\subsubsection{\texorpdfstring{\emph{Theorem 1} : Uniqueness of the
Reduced Echelon
Form}{Theorem 1 : Uniqueness of the Reduced Echelon Form}}\label{theorem-1-uniqueness-of-the-reduced-echelon-form}}

\begin{quote}
reduced chelon matrix는 유일무이하다. (row equivalent)
\end{quote}

    \hypertarget{definition-pivot-position}{%
\subsubsection{\texorpdfstring{\emph{Definition} : pivot
position}{Definition : pivot position}}\label{definition-pivot-position}}

\begin{quote}
reduced echelon form matrix에서 leading 1과 상응하는 위치, pivot
position을 포함하는 column을 pivot coloum이라고 한다.
\end{quote}

    \hypertarget{theorem-2-existence-and-uniqueness-theorem}{%
\subsubsection{\texorpdfstring{\emph{Theorem 2} : Existence and
Uniqueness
Theorem}{Theorem 2 : Existence and Uniqueness Theorem}}\label{theorem-2-existence-and-uniqueness-theorem}}

\begin{quote}
linear system이 consitent하다면,\\
(i) \textbf{unique solution} (no free variable)\\
(ii) \textbf{infinitely many solutions} (at least one free variable)
\end{quote}

\textbf{EX)}\\
* linear system\\
\(3x_{2} - 6x_{3} + 6x_{4} + 4x_{5} = -5\)\\
\(3x_{1} - 7x_{2} - 8x_{3} - 5x_{4} + 8x_{5} = 9\)\\
\(3x_{1} - 9x_{2} + 12x_{3} - 9x_{4} + 6x_{5} = 15\)

\begin{itemize}
\tightlist
\item
  augmented matrix\\
  \(\begin{bmatrix} 0 & 3 & -6 & 6 & 4 & -5 \\ 3 & -7 & -8 & -5 & 8 & 9 \\ 3 & -9 & 12 & -9 & 6 & 15\end{bmatrix}\sim \begin{bmatrix} 3 & -9 & 12 & -9 & 6 & 15 \\ 0 & 2 & -4 & 4 & 2 & -6 \\ 0 & 0 & 0 & 0 & 1 & 4 \end{bmatrix}\)
\end{itemize}

\begin{quote}
basic variables : \(x_{1}, x_{2}, x_{5}\)\\
free variables : \(x_{3}, x_{4}\)\\
따라서 이 solution은 free variable이 어떤 값을 가지냐에 따라 달라지므로
\emph{unique하지 않다.}
\end{quote}

    \begin{center}\rule{0.5\linewidth}{\linethickness}\end{center}

\hypertarget{vector-equations}{%
\subsection{1.3 Vector Equations}\label{vector-equations}}

\hypertarget{vectors-in-mathbbrn}{%
\subsubsection{\texorpdfstring{1.3.1 Vectors in
\(\mathbb{R}^n\)}{1.3.1 Vectors in \textbackslash{}mathbb\{R\}\^{}n}}\label{vectors-in-mathbbrn}}

\begin{itemize}
\tightlist
\item
  n x 1 column matrices with n entries.
\item
  represented geometrically by points in a n-dimensional coordinate
  space
\item
  Two vectors in \(\mathbb{R}^n\) are equal if and only if their
  corresponding etries are equal (because vectors are ordered pairs of
  real numbers.)
\end{itemize}

\begin{quote}
\textbf{EX)}\\
vector \textbf{u}, \textbf{v} in \(\mathbb{R}^2\) (즉, n=2) vectos with
two entries\\
\textbf{u} = \(\begin{bmatrix} 3 \\ -1 \end{bmatrix}\), \textbf{v} =
\(\begin{bmatrix} .2 \\ .3 \end{bmatrix}\)
\end{quote}

\begin{quote}
\mbox{}%
\hypertarget{algebraic-properties-of-mathbbrn}{%
\paragraph{\texorpdfstring{Algebraic Properties of
\(\mathbb{R}^n\)}{Algebraic Properties of \textbackslash{}mathbb\{R\}\^{}n}}\label{algebraic-properties-of-mathbbrn}}

for all \textbf{u, v, w} in \(\mathbb{R}^n\) and all scalars \emph{c}
and \emph{d}: 1. \textbf{u} + \textbf{v} = \textbf{v} + \textbf{u} 2.
(\textbf{u} + \textbf{v}) + \textbf{w} = \textbf{u} + (\textbf{v} +
\textbf{w}) 3. \textbf{u} + \textbf{0} = \textbf{0} + \textbf{u} =
\textbf{u} 4. \textbf{u} + (-\textbf{u}) = (-\textbf{u}) + \textbf{u} =
\textbf{0}, where -\textbf{u} denotes (-1)\textbf{u} 5.
\emph{c}(\textbf{u} + \textbf{v}) = \emph{c} \textbf{u} + \emph{c}
\textbf{v} 6. (\emph{c} + \emph{d})\textbf{u} = \emph{c} \textbf{u} +
\emph{d} \textbf{u} 7. \emph{c}(\emph{d} \textbf{u}) =
(\emph{cd})\textbf{u} 8. 1\textbf{u} = \textbf{u}
\end{quote}

    \hypertarget{definition-linear-combinations}{%
\subsubsection{\texorpdfstring{\emph{Definition} : linear
combinations}{Definition : linear combinations}}\label{definition-linear-combinations}}

\begin{quote}
if \(\mathbf{v_{1}}, \dots, \mathbf{v_{n}}\) are in \(\mathbb{R}^n\),\\
the set of all linear combinations of
\(\mathbf{v_{1}}, \dots, \mathbf{v_{n}}\)은
Span\{\(\mathbf{v_{1}}, \dots, \mathbf{v_{n}}\)\}과 동일하다. 즉,
Span\{\(\mathbf{v_{1}}, \dots, \mathbf{v_{n}}\)\}은 벡터들의 모임이며
\(c_{1}\mathbf{v_{1}} + \dots + c_{p}\mathbf{v_{p}}\) (c는 스칼라)로
표현가능하다.
\end{quote}

\begin{quote}
\textbf{같은 표현} * vector equation\\
\(x_{1}\mathbf{a_{1}} + x_{2}\mathbf{a_{2}} + \dots + x_{n}\mathbf{a_{n}} = \mathbf{b}\)\\
* augmented matrix\\
\(\begin{bmatrix} \mathbf{a_{1}} & \mathbf{a_{2}} & \dots & \mathbf{a_{n}} & \mathbf{b} \end{bmatrix}\)
\end{quote}

    \hypertarget{a-geometric-description-of-spanmathbfv-and-spanmathbfu-v}{%
\subsubsection{\texorpdfstring{1.3.2 A Geometric Description of
Span\{\(\mathbf{v}\)\} and
Span\{\(\mathbf{u, v}\)\}}{1.3.2 A Geometric Description of Span\{\textbackslash{}mathbf\{v\}\} and Span\{\textbackslash{}mathbf\{u, v\}\}}}\label{a-geometric-description-of-spanmathbfv-and-spanmathbfu-v}}

어떤 벡터가 Span한다는 것은 벡터들의 가능한 모든 linear combination으로
공간을 형성하는 것을 의미, 형성되는 공간은 조합되는 벡터에 따라
\(\mathbb{R}^n\), 또는 subspace가 될 수도 있다.

    \begin{center}\rule{0.5\linewidth}{\linethickness}\end{center}

\hypertarget{the-matrix-equation-amathbfx-mathbfb}{%
\subsection{\texorpdfstring{1.4 The Matrix Equation
\(A\mathbf{x} = \mathbf{b}\)}{1.4 The Matrix Equation A\textbackslash{}mathbf\{x\} = \textbackslash{}mathbf\{b\}}}\label{the-matrix-equation-amathbfx-mathbfb}}

\hypertarget{definition-amathbfx}{%
\subsubsection{\texorpdfstring{\emph{Definition} :
\(A\mathbf{x}\)}{Definition : A\textbackslash{}mathbf\{x\}}}\label{definition-amathbfx}}

\begin{quote}
\(A\mathbf{x} =\)
\(\begin{bmatrix} \mathbf{a_{1}} & \mathbf{a_{2}} & \dots & \mathbf{a_{n}} \end{bmatrix}\)
\(\begin{bmatrix} x_{1} \\ \dots \\ x_{n} \end{bmatrix}\)
\(= x_{1}\mathbf{a_{1}} + \dots + x_{n}\mathbf{a_{n}}\) * \(A\) : m x n
matrix with columns \$\mathbf{a_{1}}, \dots, \mathbf{a_{n}} \$ * \(x\) :
weights
\end{quote}

    \hypertarget{theorem-3}{%
\subsubsection{\texorpdfstring{\emph{Theorem
3}}{Theorem 3}}\label{theorem-3}}

\begin{itemize}
\tightlist
\item
  \(A\) : m x n matrix with columns \$\mathbf{a_{1}}, \dots,
  \mathbf{a_{n}} \$
\item
  \(\mathbf{b}\) : in \(\mathbb{R}^n\) \textgreater{} * matrix equation
  \textgreater{} \(A\mathbf{x} = \mathbf{b}\) \textgreater{} * vector
  equation \textgreater{}
  \(x_{1}\mathbf{a_{1}} + \dots + x_{n}\mathbf{a_{n}}\) \textgreater{} *
  augmented matrix \textgreater{}
  \(\begin{bmatrix} \mathbf{a_{1}} & \mathbf{a_{2}} & \dots & \mathbf{a_{n}} & \mathbf{b} \end{bmatrix}\)
\end{itemize}

\hypertarget{theorem-4-existence-of-solutions}{%
\subsubsection{\texorpdfstring{\emph{Theorem 4} : Existence of
Solutions}{Theorem 4 : Existence of Solutions}}\label{theorem-4-existence-of-solutions}}

\begin{quote}
아래 네 문장은 모두 동일한 의미이다. 1. \(\mathbb{R}^n\)에서 각
\(\mathbf{b}\)에 대해, \(A\mathbf{x} = \mathbf{b}\)는 solution을 가지고
있다. 2. \(\mathbb{R}^n\)에서 각 \(\mathbf{b}\)는 \(A\)의 column들의
linear combination이다. 3. \(A\)의 column들은 \(\mathbb{R}^n\)을
span한다. 4. \(A\)는 모든 row에서 pivot position을 갖는다. *
\emph{Warning} : coefiicient matrix에서 해당한다.
\end{quote}

\hypertarget{theorem-5-properies-of-the-matrix-vector-or-product-amathbfx}{%
\subsubsection{\texorpdfstring{\emph{Theorem 5} : Properies of the
Matrix-Vector or Product
\(A\mathbf{x}\)}{Theorem 5 : Properies of the Matrix-Vector or Product A\textbackslash{}mathbf\{x\}}}\label{theorem-5-properies-of-the-matrix-vector-or-product-amathbfx}}

\begin{quote}
\(A\)는 m x n matrix, \textbf{u, v} vectors in \(\mathbb{R}^n\), \(c\)는
scalar 1. \(A(\mathbf{u} + \mathbf{v}) = A\mathbf{u} + A\mathbf{v}\) 2.
\(A(c\mathbf{u}) = c(A\mathbf{u})\)
\end{quote}

    \begin{center}\rule{0.5\linewidth}{\linethickness}\end{center}

\hypertarget{solution-sets-of-linear-systems}{%
\subsection{1.5 Solution Sets of Linear
Systems}\label{solution-sets-of-linear-systems}}

\hypertarget{homogeneous-linear-systems-amathbfx-mathbf0}{%
\subsubsection{\texorpdfstring{1.5.1 Homogeneous Linear Systems,
\(A\mathbf{x} = \mathbf{0}\)}{1.5.1 Homogeneous Linear Systems, A\textbackslash{}mathbf\{x\} = \textbackslash{}mathbf\{0\}}}\label{homogeneous-linear-systems-amathbfx-mathbf0}}

\begin{itemize}
\tightlist
\item
  \(A\mathbf{x} = \mathbf{0}\)
\item
  항상 적어도 하나의 solution을 갖는다. \(\mathbf{x}=\mathbf{0}\)
\item
  위의 zero solution은 trivial solution이다. (;필요없는)
\item
  적어도 하나 이상의 free variable을 가질 때 non-trivial solution이다.
  \textgreater{} EX)\\
  \textgreater{} \(3x_{1} + 5x_{2} - 4x_{3} = 0\)\\
  \textgreater{} \(-3x_{1} - 2x_{2} - 8x_{3} = 0\)\\
  \textgreater{} \(6x_{1} + x_{2} - 8x_{3} = 0\)\\
  \textgreater{}
  \(\begin{bmatrix} 3 & 5 & -4 & 0 \\ -3 & -2 & -8 & 0 \\ 6 & 1 & -8 & 0\end{bmatrix}\)
  \textbf{\textasciitilde{}}
  \(\begin{bmatrix} 3 & 5 & -4 & 0 \\ 0 & 3 & 0 & 0 \\ 0 & -9 & 0 & 0\end{bmatrix}\)
  \textbf{\textasciitilde{}}
  \(\begin{bmatrix} 3 & 5 & -4 & 0 \\ 0 & 3 & 0 & 0 \\ 0 & 0 & 0 & 0\end{bmatrix}\)
  \textbf{\textasciitilde{}}
  \(\begin{bmatrix} 1 & 0 & -4/3 & 0 \\ 0 & 1 & 0 & 0 \\ 0 & 0 & 0 & 0\end{bmatrix}\)\\
  \textgreater{} * \(x_{1} - 4/3x_{3} = 0\) \textgreater{} *
  \(x_{2} = 0\) \textgreater{} * 0 = 0\\
  \textgreater{}
  \(\mathbf{x} = \begin{bmatrix} x_{1} \\ x_{2} \\ x_{3} \end{bmatrix} = \begin{bmatrix} 4/3x_{3} \\ 0 \\ x_{3} \end{bmatrix} = x_{3}\begin{bmatrix} 4/3 \\ 0 \\ 1 \end{bmatrix} = x_{3}\mathbf{v}\),
  where \(\mathbf{v}=\begin{bmatrix} 4/3 \\ 0 \\ 1 \end{bmatrix}\)
  \textgreater{} * \textbf{따라서 free variable인 \(x_{3}\)이 nontrivial
  하다.}
\end{itemize}

\hypertarget{parametric-vector-form}{%
\subsubsection{1.5.2 Parametric Vector
Form}\label{parametric-vector-form}}

\(\mathbf{x} = s\mathbf{u} + t\mathbf{v}\)

\hypertarget{solutions-of-nonhomogeneous-systems-amathbfx-mathbfb}{%
\subsubsection{\texorpdfstring{1.5.3 Solutions of Nonhomogeneous
Systems,
\(A\mathbf{x} = \mathbf{b}\)}{1.5.3 Solutions of Nonhomogeneous Systems, A\textbackslash{}mathbf\{x\} = \textbackslash{}mathbf\{b\}}}\label{solutions-of-nonhomogeneous-systems-amathbfx-mathbfb}}

\begin{quote}
EX)\\
\(A = \begin{bmatrix} 3 & 5 & -4 \\ -3 & -2 & 4 \\ 6 & 1 & -8\end{bmatrix}\)
and \(\mathbf{b} = \begin{bmatrix} 7 \\ -1 \\ -4 \end{bmatrix}\)\\
\(\begin{bmatrix} A & \mathbf{b} \end{bmatrix} = \begin{bmatrix} 3 & 5 & -4 & 7 \\ -3 & -2 & 4 & -1 \\ 6 & 1 & -8 & -4\end{bmatrix}\)
\textbf{\textasciitilde{}}
\(\begin{bmatrix} 1 & 0 & -4/3 & -1 \\ 0 & 1 & 0 & 2 \\ 0 & 0 & 0 & 0\end{bmatrix}\)\\
* \(x_{1} - 4/3x_{3} = -1\) * \(x_{2} = 2\) * 0 = 0\\
\(\mathbf{x} = \begin{bmatrix} x_{1} \\ x_{2} \\ x_{3} \end{bmatrix} = \begin{bmatrix} -1 + 4/3x_{3} \\ 2 \\ x_{3} \end{bmatrix} = \begin{bmatrix} -1 \\ 2 \\ 0 \end{bmatrix} + \begin{bmatrix} 4/3x_{3} \\ 0 \\ x_{3} \end{bmatrix} = \begin{bmatrix} -1 \\ 2 \\ 0 \end{bmatrix} + x_{3}\begin{bmatrix} 4/3 \\ 0 \\ 1 \end{bmatrix}\)\\
,\(\mathbf{x} = \mathbf{p} + x_{3}\mathbf{v}\)
\end{quote}

\includegraphics{https://user-images.githubusercontent.com/53852102/95114622-31a3ee00-077f-11eb-99a4-3c286bf52961.png}
\includegraphics{https://user-images.githubusercontent.com/53852102/95114663-3e284680-077f-11eb-845a-5fd22cd10bf6.png}

\begin{quote}
\mbox{}%
\hypertarget{parallel-solution-sets-of-amathbfx-mathbfb-and-amathbfx-mathbf0}{%
\paragraph{\texorpdfstring{Parallel solution sets of
\(A\mathbf{x} = \mathbf{b}\) and
\(A\mathbf{x} = \mathbf{0}\)}{Parallel solution sets of A\textbackslash{}mathbf\{x\} = \textbackslash{}mathbf\{b\} and A\textbackslash{}mathbf\{x\} = \textbackslash{}mathbf\{0\}}}\label{parallel-solution-sets-of-amathbfx-mathbfb-and-amathbfx-mathbf0}}

즉, \(A\mathbf{x} = \mathbf{b}\)의 solution은
\(A\mathbf{x} = \mathbf{0}\)의 solution에 \(\mathbf{p}\)를 더함으로써
얻을 수 있다.
\end{quote}

\hypertarget{theorem-6}{%
\subsubsection{\texorpdfstring{\emph{Theorem
6}}{Theorem 6}}\label{theorem-6}}

\begin{quote}
Suppose the equation \(A\mathbf{x} = \mathbf{b}\) is consistent for some
given \(\mathbf{b}\), and let \(\mathbf{p}\) be a solution. Then the
solution set of \(A\mathbf{x} = \mathbf{b}\) is the set of all vectors
of the form \(\mathbf{w} = \mathbf{p} + \mathbf{v_{h}}\), where
\(\mathbf{v_{p}}\) is solution of the homogeneous equation
\(A\mathbf{x} = \mathbf{0}\).
\end{quote}

    \begin{center}\rule{0.5\linewidth}{\linethickness}\end{center}

\hypertarget{linear-independence}{%
\subsection{1.7 Linear Independence}\label{linear-independence}}

\hypertarget{definition-linearly-independece-dependence}{%
\subsubsection{\texorpdfstring{\emph{Definition} : linearly independece
\&
dependence}{Definition : linearly independece \& dependence}}\label{definition-linearly-independece-dependence}}

\begin{quote}
indexed set of vectors \({\mathbf{v_{1}}, \dots, \mathbf{v_{p}}}\) in
\(\mathbb{R}^n\) 1. linearly independent\\
vector equation
\(x_{1}\mathbf{v_{1}} + x_{2}\mathbf{v_{2}} + \dots + x_{p}\mathbf{v_{p}} = \mathbf{0}\)가
오직 \textbf{trivial solution}만 갖는다면
\({\mathbf{v_{1}}, \dots, \mathbf{v_{p}}}\)는 Linearly independent하다.
2. linearly dependent\\
vector equation
\(c_{1}\mathbf{v_{1}} + c_{2}\mathbf{v_{2}} + \dots + c_{p}\mathbf{v_{p}} = \mathbf{0}\)에서
weight c가 모두 0이 아니면 \({\mathbf{v_{1}}, \dots, \mathbf{v_{p}}}\)는
Linearly dependent하다.
\end{quote}

\begin{quote}
\textbf{EX) set \({\mathbf{v_{1}, v_{2}, v_{3}}}\)이 linearly
independent한가?}\\
\(\mathbf{v_{1}} = \begin{bmatrix} 1 \\ 2 \\ 3 \end{bmatrix}\),
\(\mathbf{v_{2}} = \begin{bmatrix} 4 \\ 5 \\ 6 \end{bmatrix}\),
\(\mathbf{v_{3}} = \begin{bmatrix} 2 \\ 1 \\ 0 \end{bmatrix}\)\\
sol)\\
\(\begin{bmatrix} 1&4&2&0 \\ 2&5&1&0 \\ 3&6&0&0 \end{bmatrix}\)
\textbf{\textasciitilde{}}
\(\begin{bmatrix} 1&0&-2&0 \\ 0&1&1&0 \\ 0&0&0&0 \end{bmatrix}\)\\
* basic variable : \(x_{1}, x_{2}\) (= trivial variable)\\
* free variable : \(x_{3}\) (= nontrvial variable)\\
* 즉, nontrvial variable이 존재하므로 set
\({\mathbf{v_{1}, v_{2}, v_{3}}}\)은 linearly dependent하다.
\end{quote}

    \hypertarget{sets-of-one-or-two-vectors}{%
\subsubsection{1.7.1 Sets of One or Two
Vectors}\label{sets-of-one-or-two-vectors}}

\includegraphics{https://user-images.githubusercontent.com/53852102/95114754-60ba5f80-077f-11eb-8308-e0df10605139.png}
\textgreater{} * \textbf{위} :
\(\mathbf{v_{1}} = \begin{bmatrix} 3 \\ 1 \end{bmatrix}\),
\(\mathbf{v_{2}} = \begin{bmatrix} 6 \\ 2\end{bmatrix}\)\\
\textgreater{} \(\mathbf{v_{2}} = 2\mathbf{v_{1}}\), 즉 scalar
multiple로 만들어 낼 수 있다. --\textgreater{} linearly dependent\\
\textgreater{} * \textbf{아래} :
\(\mathbf{v_{1}} = \begin{bmatrix} 3 \\ 2 \end{bmatrix}\),
\(\mathbf{v_{2}} = \begin{bmatrix} 6 \\ 2\end{bmatrix}\)\\
\textgreater{} scalar multiple로 만들어 낼 수 없다. --\textgreater{}
linearly independent

\hypertarget{sets-of-two-or-more-vectors}{%
\subsubsection{1.7.2 Sets of Two or More
Vectors}\label{sets-of-two-or-more-vectors}}

\hypertarget{theorem-7-characterization-of-linearly-dependent-sets}{%
\subsubsection{\texorpdfstring{\emph{Theorem 7} : Characterization of
Linearly Dependent
Sets}{Theorem 7 : Characterization of Linearly Dependent Sets}}\label{theorem-7-characterization-of-linearly-dependent-sets}}

\begin{quote}
set \(S =\)\{\(\mathbf{v_{1}}, \dots, \mathbf{v_{p}}\)\}의 둘 또는 그
이상의 벡터가 linearly dependent하다면 set \(S\)의 벡터로 다른 벡터를
linear combination으로 표현 가능하다.
\end{quote}

\textbf{EX) \(\mathbf{u} = \begin{bmatrix} 3\\1\\0 \end{bmatrix}\),
\(\mathbf{v} = \begin{bmatrix} 1\\6\\0 \end{bmatrix}\)가 주어졌다.
\{\(\mathbf{u, v, w}\)\}가 linearly dependent할 때,
Span\{\(\mathbf{u, v}\)\}에서 벡터 \(\mathbf{w}\)가 dependent한지
independent한지 설명하라.}\\
\textgreater{} \textbf{Sol)}\\
\textgreater{} \{\(\mathbf{u, v, w}\)\}가 linearly dependent하다는 것은
\textgreater{} * Span\{\(\mathbf{u, v}\)\}와
Span\{\(\mathbf{u, v, w}\)\}이 동일 \textgreater{} * \(\mathbf{w}\)가
\(\mathbf{u}\), \(\mathbf{v}\)의 linear combination으로 형성 가능
\textgreater{} * \(\mathbf{w}\)가 Span\{\(\mathbf{u, v}\)\}에
속해있다.\\
\textgreater{} 는 뜻이다. 따라서 Span\{\(\mathbf{u, v}\)\}에서 벡터
\(\mathbf{w}\)가 dependent하다.

\begin{quote}
\begin{figure}
\centering
\includegraphics{https://user-images.githubusercontent.com/53852102/95114800-72036c00-077f-11eb-900a-2164df860d3c.png}
\caption{theorem7}
\end{figure}
\end{quote}

    \hypertarget{theorem-8}{%
\subsubsection{\texorpdfstring{\emph{Theorem
8}}{Theorem 8}}\label{theorem-8}}

\begin{quote}
\(p>n\)일 때, set \{\(\mathbf{v_{1}}, \dots, \mathbf{v_{p}}\)\} in
\(\mathbb{R}^n\)은 lineearly dependent하다.
\end{quote}

\hypertarget{theorem-9}{%
\subsubsection{\texorpdfstring{\emph{Theorem
9}}{Theorem 9}}\label{theorem-9}}

\begin{quote}
set \(S =\)\{\(\mathbf{v_{1}}, \dots, \mathbf{v_{p}}\)\} in
\(\mathbb{R}^n\)이 zero vector를 포함하고 있다면 그 set은 linearly
dependent하다.
\end{quote}

    \begin{center}\rule{0.5\linewidth}{\linethickness}\end{center}

\hypertarget{introduction-to-linear-transformations}{%
\subsection{1.8 Introduction to Linear
Transformations}\label{introduction-to-linear-transformations}}

\hypertarget{linear-transformations}{%
\subsubsection{1.8.1 Linear
Transformations}\label{linear-transformations}}

\hypertarget{a-transformationor-function-or-mapping-t-from-mathbbrn-to-mathbbrm-is-a-rule-that-assigns-to-each-vector-mathbfx-in-mathbbrn-a-vector-tmathbfx-in-mathbbrm}{%
\paragraph{\texorpdfstring{A transformation(or function or mapping)
\(T\) from \(\mathbb{R}^n\) to \(\mathbb{R}^m\) is a rule that assigns
to each vector \(\mathbf{x}\) in \(\mathbb{R}^n\) a vector
\(T(\mathbf{x})\) in
\(\mathbb{R}^m\)}{A transformation(or function or mapping) T from \textbackslash{}mathbb\{R\}\^{}n to \textbackslash{}mathbb\{R\}\^{}m is a rule that assigns to each vector \textbackslash{}mathbf\{x\} in \textbackslash{}mathbb\{R\}\^{}n a vector T(\textbackslash{}mathbf\{x\}) in \textbackslash{}mathbb\{R\}\^{}m}}\label{a-transformationor-function-or-mapping-t-from-mathbbrn-to-mathbbrm-is-a-rule-that-assigns-to-each-vector-mathbfx-in-mathbbrn-a-vector-tmathbfx-in-mathbbrm}}

\begin{itemize}
\tightlist
\item
  \(T\) is m x n
\item
  set \(\mathbb{R}^n\) : \textbf{domain} of \(T\) ; 정의역
\item
  set \(\mathbb{R}^m\) : \textbf{codomain} of \(T\) ; 공역
\item
  set of all images \(T(\mathbf{x})\) : \textbf{range} of \(T\)\\
  \includegraphics{https://user-images.githubusercontent.com/53852102/95114849-88a9c300-077f-11eb-80f6-457b423c8f6a.png}
\end{itemize}

\hypertarget{definition}{%
\subsubsection{\texorpdfstring{\emph{Definition}}{Definition}}\label{definition}}

\begin{quote}
A transformation(or mapping) \(T\) is linear if:\\
1. \(T(\mathbf{u + v}) = T(\mathbf{u}) + T(\mathbf{v})\) for all
\(\mathbf{u, v}\) in the domain of \(T\). ; additivity\\
2. \(T(c\mathbf{u}) = cT(\mathbf{u})\) for all scalars \(c\) an all
\(\mathbf{u}\) in the domain of \(T\). ; homogeneity
\end{quote}

\begin{verbatim}
> * following useful facts
> 3. $T(\mathbf{0}) = \mathbf{0}$  
> 2에 의해 $T(\mathbf{0}) = T(0\mathbf{u}) = 0T(\mathbf{u}) = \mathbf{0}$
> 4. $T(c\mathbf{u} + d(\mathbf{v}) = cT(\mathbf{u}) + dT(\mathbf{v})$  
> 1과 2에 의해 $T(c\mathbf{u} + d(\mathbf{v}) = T(c\mathbf{u}) + T(d\mathbf{v}) = cT(\mathbf{u}) + dT(\mathbf{v})$
\end{verbatim}

\begin{itemize}
\tightlist
\item
  \textbf{superposition principle}\\
  \(T(c_{1}\mathbf{v_{1}} + \dots + c_{p}\mathbf{v_{p}}) = c_{1}T(\mathbf{v_{1}}) + \dots + c_{p}T(\mathbf{v_{p}})\)\\
  The system satisfies the superposition principle if whenever an input
  is expressed as a linear combination of such signals, the system's
  response is the same linear combination of the responses to the
  individual signals.
\end{itemize}

    \begin{center}\rule{0.5\linewidth}{\linethickness}\end{center}

\hypertarget{the-matrix-of-a-linear-transformation}{%
\subsection{1.9 The Matrix of a Linear
Transformation}\label{the-matrix-of-a-linear-transformation}}

    \hypertarget{theorem-10}{%
\subsubsection{\texorpdfstring{\emph{Theorem
10}}{Theorem 10}}\label{theorem-10}}

\begin{quote}
Let \(T : \mathbb{R}^n {\rightarrow} \mathbb{R}^m\) be a linear
transformation. Then there \textbf{exists a unique matrix} \(A\) such
that\\
\[T(\mathbf{x}) = A\mathbf{x}\] for all \(\mathbf{x}\) in
\(\mathbb{R}^n\)\\
In fact, \(A\) is the \(m * n\) matrix whose \(j\)th column is the
vector \(T(\mathbf{e_{j}})\), where \(\mathbf{e_{j}}\) is the \(j\)th
column of the identity matrix in \(\mathbb{R}^n\):\\
\[A = \begin{bmatrix} T(\mathbf{e_{1}}) & \dots & T(\mathbf{e_{j}}) \end{bmatrix}\]
(; \(A\) = \textbf{standard matrix for the linear transformation \(T\)})
\end{quote}

    \hypertarget{definition-onto}{%
\subsubsection{\texorpdfstring{\emph{Definition} :
onto}{Definition : onto}}\label{definition-onto}}

\begin{quote}
A mapping \(T : \mathbb{R}^n {\rightarrow} \mathbb{R}^m\) is said to be
\textbf{onto} \(\mathbb{R}^m\) if each \textbf{b} in \(\mathbb{R}^m\) is
the image of at least one \(\mathbf{x}\) in \(\mathbb{R}^n\)\\
\includegraphics{https://user-images.githubusercontent.com/53852102/95114915-9eb78380-077f-11eb-93b9-d0ca4af902d7.png}
\end{quote}

    \hypertarget{definition-one-to-one}{%
\subsubsection{\texorpdfstring{\emph{Definition} :
one-to-one}{Definition : one-to-one}}\label{definition-one-to-one}}

\begin{quote}
A mapping \(T : \mathbb{R}^n {\rightarrow} \mathbb{R}^m\) is said to be
\textbf{one-to-one} if each \textbf{b} in \(\mathbb{R}^m\) is the image
of at most one \(\mathbf{x}\) in \(\mathbb{R}^n\)\\
\includegraphics{https://user-images.githubusercontent.com/53852102/95114953-ae36cc80-077f-11eb-8fb4-7607d44bf909.png}
\end{quote}

    \hypertarget{theorem-11}{%
\subsubsection{\texorpdfstring{\emph{Theorem
11}}{Theorem 11}}\label{theorem-11}}

\begin{quote}
Let \(T : \mathbb{R}^n {\rightarrow} \mathbb{R}^m\) be a linear
transformation. Then \(T\) is one-to-one if and only if the equation
\(T(\mathbf{x}) = \mathbf{0}\) has \textbf{only trivial solution.}
\end{quote}

    \hypertarget{theorem-12}{%
\subsubsection{\texorpdfstring{\emph{Theorem
12}}{Theorem 12}}\label{theorem-12}}

\begin{quote}
Let \(T : \mathbb{R}^n {\rightarrow} \mathbb{R}^m\) be a linear
transformation. and let \(A\) be the standard matrix for \(T\). then:\\
a. \(T\) maps \(\mathbb{R}^n\) onto \(\mathbb{R}^m\) if and only if the
columns of \(A\) span \(\mathbb{R}^m\)\\
b. \(T\) is one-to-one if and only if the columns of \(A\) are linearly
independent.
\end{quote}

    \begin{tcolorbox}[breakable, size=fbox, boxrule=1pt, pad at break*=1mm,colback=cellbackground, colframe=cellborder]
\prompt{In}{incolor}{ }{\boxspacing}
\begin{Verbatim}[commandchars=\\\{\}]

\end{Verbatim}
\end{tcolorbox}


    % Add a bibliography block to the postdoc
    
    
    
\end{document}
